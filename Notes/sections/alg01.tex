 \subsection{Proposed GSVD algorithm} \label{alg}
    The algorithm we propose consists of four steps. First is the pre-processing step when we reduce the input matrix pair to a triangular pair while revealing their ranks. \cite{bai1993new} We further reduce two upper triangular matrices to one upper triangular matrix in the QR decomposition step. Next is the CS decomposition of a matrice with orthonormal columns that is partitioned into two blocks. \cite{van1976generalizing} The last step is post-processing to get the final product of the decomposition. 
    
    \begin{enumerate}[\textit{Step} 1]
        \item Pre-processing: 
        
            To reduce regular matrices to their triangular form and reveal rank, we employ URV decomposition (QR decomposition with column pivoting followed by RQ decomposition) \cite{golub2013matrix} as well as QR decomposition. We detail this in nine steps below.
            
            \begin{enumerate}[(1)]
                \item QR decomposition with column pivoting of $B$:
                \begin{displaymath}
                    BP = V\bordermatrix{ & l & n-\ell \cr
                    \hfill \ell & B_{11} & B_{12} \cr
                    \hfill p-\ell & 0 & 0 \cr}
                \end{displaymath}
                \item Update $A: \ A = AP$
                \item Set $Q: \  Q = I, \ \ Q = QP$
                \item If $p \geq \ell$ and $n \neq \ell$:
                    \begin{itemize}
                        \item RQ decomposition of $(B_{11} \ \ B_{12})$:
                            \begin{displaymath}
                                \bordermatrix{ & \ell & n-\ell \cr
                                \hfill \ell & B_{11} & B_{12} \cr}
                                = \bordermatrix{ & n-\ell & \ell \cr
                                \hfill \ell & 0 & B_{13} \cr}Z
                            \end{displaymath}
                        \item Update $A: \ A = AZ^{T}$
                        \item Update $Q: \ Q = QZ^{T}$
                    \end{itemize}
                \item Let
                    \begin{displaymath}
                    A = \bordermatrix{ & n-\ell & \ell \cr
                            \hfill m & A_{1} & A_{2} \cr},
                    \end{displaymath} 
                    then QR decomposition with column pivoting of $A_1$ is:
                     \begin{displaymath}
                        A_{1}P_{1} = U\bordermatrix{ & k & n-\ell-k \cr
                        \hfill k & A_{11} & A_{12} \cr
                        \hfill m-k & 0 & 0 \cr}
                    \end{displaymath}
                \item Update $A_{2}: \ A_{2} = U^{T}A_{2}$
                \item Update $Q: \ Q[1:n, 1:n-\ell] = Q[1:n, 1:n-\ell]P_{1}$
                \item If $n-\ell \geq k$:
                    \begin{itemize}
                        \item RQ decomposition of $(A_{11} \ \ A_{12})$:
                            \begin{displaymath}
                                \bordermatrix{ & k & n-\ell-k \cr
                                \hfill k & A_{11} & A_{12} \cr}
                                = \bordermatrix{ & n-\ell-k & k \cr
                                \hfill k & 0 & A_{12} \cr}Z_{1}
                            \end{displaymath}
                        \item Update $Q: \ Q[1:n, 1:n-\ell] = Q[1:n, 1:n-\ell]Z_{1}^{T}$
                    \end{itemize}
                \item If $m \geq k$:
                    Let 
                    \begin{displaymath}
                    A_{2} = \bordermatrix{ & \ell  \cr
                            \hfill k & A_{13}  \cr
                            \hfill m-k & A_{23} \cr}
                    \end{displaymath} 
                    \begin{itemize}
                        \item QR decomposition of $A_{23}$:
                        \begin{displaymath}
                        A_{23} = U_{1}\bordermatrix{ & \ell  \cr
                            \hfill \ell & A_{23}  \cr
                            \hfill m-k-\ell & 0 \cr}
                    \end{displaymath} 
                        \item Update $U: \ U[:,k+1:m] = U[:,k+1:m]U_{1}$
                    \end{itemize}    
            \end{enumerate}
            
            Putting it together, we have the following decomposition as pre-processing: 
            \begin{equation} \label{eq-alg-1}
                A = UR_{A}Q^{T},\ \ \ \ B = VR_{B}Q^{T}
            \end{equation}
            where
            \begin{displaymath}
                R_{A} = \bordermatrix{ & n-k-\ell & k & \ell \cr
                \hfill k & 0 & A_{12} & A_{13} \cr
                \hfill \ell & 0 & 0 & A_{23} \cr
                \hfill m-k-\ell & 0 & 0 & 0}, \  \ \ \
                R_{B} = \bordermatrix{ & n-k-\ell & k & \ell   \cr
                \hfill \ell & 0 & 0 & B_{13}\cr
                \hfill p-\ell & 0 & 0 & 0}
            \end{displaymath}
            overwrite $A$ and $B$, respectively, and $A_{12}$ and $B_{13}$ are non-singular upper triangular matrix. $\ell$ is the rank of $B$, $k+\ell$ is the rank of $[A^T \ B^T]^T$. If $m-k-\ell \geq 0$, $A_{23}$ is $\ell$-by-$\ell$ upper triangular, otherwise, it's $(m-k)$-by-$\ell$ upper trapezoidal. 
        
        \item QR decomposition of $[A_{23}^{T} \ B_{13}^{T}]^T$:
            \begin{displaymath}
                    \bordermatrix{ & \ell  \cr
                        \hfill \ell & A_{23}  \cr
                        \hfill \ell & B_{23} \cr} = 
                    \bordermatrix{ & \ell  \cr
                        \hfill \ell & Q_{1}  \cr
                        \hfill \ell & Q_{2} \cr}R_{23}   
            \end{displaymath}
            Thus, \eqref {eq-alg-1} can be rewritten as:
            \begin{equation} \label{eq-alg-2}
                A = U(Q_{A}\hat{R})Q^{T}, \ \ \ \ B = V(Q_{B}\hat{R})Q^{T}
            \end{equation}
            where 
            \begin{displaymath}
                Q_{A} = \bordermatrix{ & k & \ell \cr
                \hfill k & I & 0 \cr
                \hfill \ell & 0 & Q_1 \cr
                \hfill m-k-\ell & 0 & 0}, \  \ \ \
                Q_{B} = \bordermatrix{ & k & \ell   \cr
                \hfill \ell & 0 & Q_2\cr
                \hfill p-\ell & 0 & 0}, \ \ \ \
                \hat{R} = \bordermatrix{ & n-k-\ell & k & \ell \cr
                \hfill k & 0 & A_{12} & B_{13} \cr
                \hfill \ell & 0 & 0 & R_{23}}
            \end{displaymath}
        
            If $m-k-\ell \geq 0$, $Q_1$ is $\ell$-by-$\ell$, otherwise, $Q_1$ is $(m-k)$-by-$\ell$.
        
        \item CS decomposition of $Q_1$ and $Q_2$:
            \begin{align}
                Q_1 = U_1C_1Z_1^{T}, \ \ \ \ Q_2 = V_1S_1Z_1^{T}
            \end{align}
            We then can derive from \textit{Step} 2 and the above CS decomposition that 
            \begin{equation} \label{eq-alg-3}
                A = U(\hat{U}C\hat{Q}^{T})\hat{R}Q^{T}, \ \ \ \ B = V(\hat{V}S\hat{Q}^{T})\hat{R}Q^{T}
            \end{equation}
            where
            \begin{displaymath}
                \hat{U} = \bordermatrix{ & k & \ell & m-k-\ell\cr
                \hfill k & I & 0 & 0\cr
                \hfill \ell & 0 & U_1 & 0\cr
                \hfill m-k-\ell & 0 & 0 & I}, \  \ \ \
                \hat{V} = \bordermatrix{ & \ell & p-\ell   \cr
                \hfill \ell & V_1 & 0\cr
                \hfill p-\ell & 0 & I}, \ \ \ \
                \hat{Q}^{T} = \bordermatrix{ & k & \ell   \cr
                \hfill \ell & I & 0\cr
                \hfill p-\ell & 0 & Z_1^{T}}
            \end{displaymath}
            and 
            \begin{displaymath}
                C = \bordermatrix{ & k & \ell \cr
                \hfill k & I & 0 \cr
                \hfill \ell & 0 & C_1 \cr
                \hfill m-k-\ell & 0 & 0}, \  \ \ \
                S = \bordermatrix{ & k & \ell   \cr
                \hfill \ell & 0 & S_1\cr
                \hfill p-\ell & 0 & 0}
            \end{displaymath}
            Note that when $m-k-\ell < 0$, $U_1$ and $C_1$ will only have $m-k$ rows.
        
        More details regarding CS decomposition can be found in Section \ref{csd}. 
        % \cite{stewart1982computing}
        
        \item Post-processing:
            \begin{itemize}
                \item $U = U \hat{U}$.
                \item $V = V \hat{V}$.
                \item Formulate $R$ by RQ decomposition: $\hat{Q}^{T}\hat{R} = RQ_{3}$ 
                \item $Q = Q Q_{3}^{T}$
            \end{itemize}
    \end{enumerate}
    
    To sum up, we can obtain:
        \begin{equation} \label{eq-alg-4}
            A = UCRQ^{T}, \ \ \ \ B = VSRQ^{T}
        \end{equation}
    
    $C$ and $S$ have the following structures:
    
    \begin{itemize}
        \item if $m \ge k+\ell$
            \begin{displaymath}
                C = \bordermatrix{ & k & \ell  \cr
                \hfill k & I & 0 \cr
                \hfill \ell & 0 & \Sigma_1 \cr
                m-k-\ell & 0 & 0}, \  \ \ \
                S = \bordermatrix{ & k & \ell \cr
                \hfill \ell & 0 & \Sigma_2 \cr
                \hfill p-\ell & 0 & 0}
            \end{displaymath}
                
        \item if $m < k+\ell$
            \begin{displaymath}
                C = \bordermatrix{ & k & m-k & k+\ell-m  \cr
                \hfill k & I & 0 & 0\cr
                \hfill m-k & 0 & \Sigma_1 & 0}, \  \ \ \
                S = \bordermatrix{ & k & m-k & k+\ell-m \cr
                \hfill m-k & 0 & \Sigma_2 & 0\cr
                \hfill k+\ell-m & 0 & 0 & I\cr
                \hfill p-\ell & 0 & 0 & 0}
            \end{displaymath}
    \end{itemize}
    
    In either case, $\Sigma_1^2 + \Sigma_2^2 = I$.
    
    \paragraph{Remark}
        Michael Stewart in his paper \cite{stewart2016rank} describes an alternate rank revealing mechanism of $[A; B]$ that more reliably determines the partitioning of a GSVD and shows improved numerical reliability.
    
    \subsubsection{CS Decomposition} \label{csd}
    \paragraph{Definition} Suppose we have an $(m+p)-by-n$ matrix $Q$ such that $m+p \geq n$ and has orthonormal columns. If we partition $Q$ into 2-by-1 form as $[Q_1; Q_2]$, then the CS decomposition of $Q_1$ and $Q_2$ is the following:
    \begin{align}
        Q_1 = UCZ^T,\ \  \ \ Q_2 = VSZ^T
    \end{align}
    \begin{itemize}
        \item $U$ is an $m$-by-$m$ orthogonal matrix,
        \item $V$ is a $p$-by-$p$ orthogonal matrix, 
        \item $Z$ is an $n$-by-$n$ orthogonal matrix, 
        \item $C$ is an $m$-by-$n$ real, non-negative diagonal matrix,
        \item $S$ is a $p$-by-$n$ real, non-negative matrix whose top right block is diagonal,
        \item $C^{T}C + S^{T}S = I$. Write $C^{T}C = diag(\alpha_1^{2}, \alpha_2^{2}, \cdots, \alpha_n^{2})$ and $S^{T}S = diag(\beta_1^{2}, \beta_2^{2}, \cdots, \beta_n^{2})$, we have 
        \begin{align} \label{cosine-sine}
            \alpha_i^{2} + \beta_i^{2} = 1 \ \  \text{for} \ \ i = 1,2,\cdots,n
        \end{align}
    \end{itemize}
    
    Specifically, $C$ and $S$ belong to one of the four cases depending on the dimension of $Q$.
    \begin{enumerate}
        \item $m \geq n$ and $p \geq n$:
        \begin{displaymath}
            C = \bordermatrix{ & n  \cr
            \hfill n & \Sigma_1 \cr
            \hfill m-n & 0}, \  \ \ \
            S = \bordermatrix{ & n \cr
            \hfill n & \Sigma_2 \cr
            \hfill p-n & 0}
        \end{displaymath}
        
        \item $m \geq n$ and $p < n$:
        \begin{displaymath}
            C = \bordermatrix{ & n-p & p  \cr
            \hfill n-p & I & 0 \cr
            \hfill p & 0 & \Sigma_1 \cr
            \hfill m-n & 0 & 0}, \  \ \ \
            S = \bordermatrix{ & n-p & p\cr
            \hfill p & 0 & \Sigma_2}
        \end{displaymath}
        
        \item $m \leq n$ and $p \geq n$:
        \begin{displaymath}
            C = \bordermatrix{ & m & n-m\cr
            \hfill m & \Sigma_1 & 0}, \  \ \ \
            S = \bordermatrix{ & m & n-m  \cr
            \hfill m & \Sigma_2 & 0 \cr
            \hfill n-m & 0 & I \cr
            \hfill p-n & 0 & 0}
        \end{displaymath}
        
        \item $m \leq n$ and $p < n$:
        \begin{displaymath}
            C = \bordermatrix{ & n-p & t & n-m \cr
            \hfill n-p & I & 0 & 0 \cr
            \hfill t & 0 & \Sigma_1 & 0}, \  \ \ \
            S = \bordermatrix{ & n-p & t & n-m \cr
            \hfill t & 0 & \Sigma_2 & 0 \cr
            \hfill n-m & 0 & 0 & I}
        \end{displaymath}
        where $t = m+p-n$.
    \end{enumerate}
    
    Note that $\Sigma_1$ and $\Sigma_2$ in all four cases are diagonal matrices and satisfy $\Sigma_1^2 + \Sigma_2^2 = I$.

    CS decomposition is named after cosine and sine due to the resemblance between \eqref{cosine-sine} and cosine-sine relation. Thus, we name $\alpha_i$ and $\beta_i$ cosine and sine values, respectively. To align with the growth of cosine and sine values between angles of 0 and $\frac{\pi}{2}$ in Euclidean geometry, $\alpha_i$ are placed in non-increasing order while $\beta_i$ are sorted in non-decreasing order. 

    
    \paragraph{Algorithm} Now, we present the algorithm to compute the CS decomposition which extends the algorithm developed by Van Loan in \cite{vanloan85}.    
    
    First, we set $q_1 = min\{m, n\}$ and $q_2 = min\{p, n\}$. We split this algorithm into two cases: (1). $m \leq p$ and (2). $m > p$.
    
    \begin{enumerate}
        \item If $m \leq q$:
            \begin{enumerate}[\textit{Step} 1.]
                \item SVD of $Q_2$ such that:
                    \begin{equation}
                        Q_2 = VSZ^{T}
                    \end{equation}
                    $V$ is $p$-by-$p$, $Z$ is $n$-by-$n$, both are orthogonal matrices. $S$ has the following structure:
                    \begin{displaymath}
                        S = \bordermatrix{ & q_2 & n-q_2 \cr
                            \hfill q_2 & \Sigma & 0 \cr
                            \hfill p-q_2 & 0 & 0}
                    \end{displaymath}
                    where $\Sigma = diag(\beta_n, \cdots, \beta_{n-q_2+1})$ such that $1 \geq \beta_n \geq \beta_{n-1} \geq \cdots \geq \beta_{n-q_2+1} \geq 0$. This means we need to reverse the ordering of $\beta_{i}$ to preserve that the sine values are in non-decreasing order. Thus,
                    \begin{itemize}
                        \item Reorder the diagonal entries of $S$ in non-decreasing, such that:
                        \begin{displaymath}
                            S = \bordermatrix{ & n-q_2 & q_2 \cr
                                \hfill q_2 & 0 & \hat{\Sigma} \cr
                                \hfill p-q_2 & 0 & 0}
                        \end{displaymath} where $\hat{\Sigma} = diag(\beta_{n-q_2+1}, \cdots, \beta_{n})$
                        \item Reverse the first $q_2$ columns of $V$: $V[:,1:q_2] = V[:,q_2:-1:1]$. 
                        \item Reverse the columns of $Z$: $Z = Z[:,n:-1:1]$. 
                    \end{itemize}
                    
                    Since $Q_2$ has $(n-q_2)$ zero singular values, $\beta_{1} = \beta_{2} = \cdots = \beta_{n-q_2} = 0$ and correspondingly, $\alpha_1 = \alpha_2 = \cdots = \alpha_{n-q_2} = 1$.
                \item Determine $r$ such that $0 \leq \beta_{n-q_2+1} \leq \cdots \leq \beta_{r} \leq \frac{1}{\sqrt{2}} \leq \beta_{r+1} \leq \cdots \leq \beta_{n} \leq 1$. 
                
                (\textcolor{red}{Footnote needed to justify the choice of $\frac{1}{\sqrt{2}}$ as the threshold. Any suggestion on formatting?})
                
                \item $T = Q_1Z$.
                \item QR decomposition of $T$:
                    \begin{equation}
                        T = UR,
                    \end{equation}
                    where $U$ is an $m$-by-$m$ orthogonal matrix,
                    \begin{displaymath}
                        R = \bordermatrix{ & n-q_2 & r-n+q_2 & q_1-r & n-q_1 \cr
                                    \hfill n-q_2 & I & \epsilon & \epsilon & \epsilon \cr
                                    \hfill r-n+q_2 & 0 & R_{22} & \epsilon & \epsilon \cr
                                    \hfill q_1-r & 0 & 0 & R_{33} & R_{34} \cr
                                    \hfill m-q_1 & 0 & 0 & 0 & 0}
                    \end{displaymath}
                    and $R_{22} = diag(\alpha_{n-q_2+1}, \cdots, \alpha_{r})$.
                    
                    (\textcolor{red}{Footnote needed to justify the near-zero block matrices in the upper diagonal. This justification could be really long, where is the proper place to put it?} ) 
                    
                    Combining \textit{Step} 3 and \textit{Step} 4, we obtain:
                    \begin{align} \label{eq-q_1-case1}
                        Q_1 = URZ^{T}
                    \end{align}
                    
                    The formula above can be treated as the SVD of $Q_1$. Thus, the fact that $Q_1$ has $(n-q_1)$ zero singular values implies that $\alpha_{q_1} = \cdots = \alpha_{l} = 0$, and $\beta_{q_1} = \cdots = \beta_{l} = 1$, respectively. 
                \item SVD of $(R_{33} \ R_{34})$ such that:
                    \begin{align} \label{eq-r-svd}
                        (R_{33} \ R_{34}) = U_{r}C_{r}Z_{r}^{T}
                    \end{align}   
                    where $U_{r}$ is a $(q_1-r)$-by-$(q_1-r)$ orthogonal matrix, $Z_{r}$ is an $(n-r)$-by-$(n-r)$ orthogonal matrix and $C_{r}$ is a $(q_1-r)$-by-$(n-r)$ matrix with the main diagonal entries storing non-zero $\alpha_{r+1}, \cdots, \alpha_{q_1}$.
                    \item To plug \eqref{eq-r-svd} into \eqref{eq-q_1-case1}, we shall update $U$, $R$ and $Z$ accordingly:
                        \begin{itemize}
                            \item Update the $(r+1)$ to $q_1$ columns of $U$:
                                \begin{displaymath}
                                    U = U\bordermatrix{ & r & q_1-r & m-q_1 \cr
                                    \hfill r & I & 0 & 0 \cr
                                    \hfill q_1-r & 0 & U_{r} & 0 \cr
                                    \hfill m-q_1 & 0 & 0 & I}
                                \end{displaymath}
                            \item Update the last $(n-r)$ columns of $Z$:
                                \begin{displaymath}
                                    Z = Z\bordermatrix{ & r & n-r \cr
                                    \hfill r & I & 0 \cr
                                    \hfill n-r & 0 & Z_{r}}
                                \end{displaymath}
                            \item Rewrite $R$ to formulate $C$:
                                \begin{displaymath}
                                    C = \bordermatrix{ & n-q_2 & r-n+q_2 & q_1-r & n-q_1 \cr
                                    \hfill n-q_2 & I & 0 & 0 & 0 \cr
                                    \hfill r-n+q_2 & 0 & R_{22} & 0 & 0 \cr
                                    \hfill q_1-r & 0 & 0 & C_{r}[:,1:q_1-r] & 0 \cr
                                    \hfill m-q_1 & 0 & 0 & 0 & 0}
                                \end{displaymath}
                        \end{itemize}
                    
                    Now, we have the final decomposition of $Q_1$:
                    \begin{align}
                        Q_1 = UCZ^{T}
                    \end{align}
                    \item Since $Z$ is updated, we need to modify $V$ as well:
                        \begin{itemize}
                            \item Set $W$:
                            
                                Let $S_1 = diag(\beta_{r+1}, \cdots, \beta_{q_2})$, $W = S_1Z_{r}[1:q_2-r,1:q_2-r]$.
                            \item QR decomposition of $W$:
                                \begin{align}
                                    W = Q_{w}R_{w}
                                \end{align}
                            \item Update middle $(q_2-r)$ columns of $V$:
                            
                                Let $l = min\{r, n-q_2\}$,
                                \begin{displaymath}
                                    V = V\bordermatrix{ & r-l & q_2-r & p-q_2+l \cr
                                    \hfill r-l & I & 0 & 0 \cr
                                    \hfill q_2-r & 0 & Q_{w} & 0 \cr
                                    \hfill p-q_2+l & 0 & 0 & I}
                                \end{displaymath}
                        \end{itemize}
            \end{enumerate}
            To summarize, we obtain:
            \begin{align}
                Q_1 = UCZ^{T}, \ \ \ \ Q_2 = VSZ^{T}
            \end{align}
            and
            \begin{displaymath}
                C = \bordermatrix{ & n-q_2 & t & n-q_1 \cr
                \hfill n-q_2 & I & 0 & 0 \cr
                \hfill t & 0 & \Sigma_1 & 0 \cr
                \hfill m-q_1 & 0 & 0 & 0}, \  \ \ \
                S = \bordermatrix{ & n-q_2 & t & n-q_1 \cr
                \hfill t & 0 & \Sigma_2 & 0 \cr
                \hfill n-m & 0 & 0 & I \cr
                \hfill p-q_2 & 0 & 0 & 0}
        \end{displaymath}
        where $t = q_1 + q_2 - n$, $\Sigma_1 = diag(\alpha_{l-q_2+1}, \cdots, \alpha_{q_1})$ and $\Sigma_2 = diag(\beta_{l-q_2+1}, \cdots, \beta_{q_1})$.
        
        \item If $m > q$:
            \begin{enumerate}[\textit{Step} 1.]
                \item Full SVD of $Q_1$ such that:
                    \begin{align}
                        Q_1 = UCZ^{T}
                    \end{align}
                    $U$ is $m$-by-$m$, $Z$ is $n$-by-$n$, both are orthogonal matrices. $C$ is $m$-by-$n$ with singular values $1 \geq \alpha_1 \geq \cdots \geq \alpha_{q_1} \geq 0$ placed in the main diagonal. Since $Q_1$ has $(n-q_1)$ zero singular values, we obtain $\alpha_{q_1+1} = \cdots = \alpha_{n} = 0$, and $\beta_{q_1+1} = \cdots = \beta_{n} = 1$, respectively. 
                \item Determine $r$ such that $1 \geq \alpha_{1} \geq \cdots \geq \alpha_{r} \geq \frac{1}{\sqrt{2}} \geq \alpha_{r+1} \geq \cdots \geq \alpha_{n} \geq 0$. 
                \item $T = Q_2Z$.
                \item QL decomposition of $T$:
                    \begin{align}
                        T = VL,
                    \end{align}
                    where $V$ is a $p$-by-$p$ orthogonal matrix, 
                    \begin{displaymath}
                        L = \bordermatrix{ & n-q_2 & r & q_1+q_2-n-r & n-q_1 \cr
                                    \hfill p-q_2 & 0 & 0 & 0 & 0 \cr
                                    \hfill r & L_{11} & L_{12} & 0 & 0 \cr
                                    \hfill q_1+q_2-n-r & \epsilon & \epsilon & L_{23} & 0 \cr
                                    \hfill n-q_1 & \epsilon & \epsilon & \epsilon & I}
                    \end{displaymath}
                    and $L_{23} = diag(\beta_{n-q_2+r+1}, \cdots, \beta_{q_1})$.
                
                    To be consistent with the structure of $S$ given above, we pre-multiply $T$ with a permutation matrix $P$ in an effort to move the top $(n-q_2)$ rows to the bottom.
                    \begin{displaymath}
                        P = \bordermatrix{& p-q_2 & r & q_2-r \cr
                            \hfill r & 0 & I & 0 \cr
                            \hfill q_2-r & 0 & 0 & I \cr
                            \hfill p-q_2 & I & 0 & 0}
                    \end{displaymath}
                    Combining \textit{Step} 3 and \textit{Step} 4, we get:
                        \begin{align} \label{eq-q_2-case2}
                            Q_2 = V(P^{-1}PL)Z^{T}
                        \end{align}
                    This formula can be regarded as the SVD of $Q_2$. Therefore, the fact that $Q_2$ has $(n-q_2)$ zero singular values indicates that $\alpha_{1} = \cdots = \alpha_{n-q_2} = 1$, and $\beta_{1} = \cdots = \beta_{n-q_2} = 0$, respectively. 
                \item SVD of $(L_{11} \ L_{12})$ such that:
                    \begin{align} \label{eq-l-svd}
                        (L_{11} \ L_{12}) = V_{l}S_{l}Z_{l}^{T}
                    \end{align}
                    where $V_{l}$ is $r$-by-$r$ orthogonal matrix, $Z_{l}$ is $(n-q_2+r)$-by-$(n-q_2+r)$ orthogonal matrix and $S_{l}$ is $r$-by-$(n-q_2+r)$ and contains the $r$ singular values in a non-increasing fashion. However, by the nature of sine, we want to reverse the ordering of $\beta_i$. Accordingly, we need to reverse the columns of $V_{l}$ and $Z_{l}$.
                        \begin{itemize}
                            \item Reorder the diagonal entries of $S_{l}$ in non-decreasing order, such that:
                            \begin{displaymath}
                                S_{l} = \bordermatrix{ & r & n-q_2\cr
                                    \hfill r & \Sigma & 0}
                            \end{displaymath} where $\Sigma = diag(\beta_{n-q_2+1}, \cdots, \beta_{n-q_2+r})$
                            \item Reverse the columns of $V_l$: $V_l = V_l[:,r:-1:1]$. 
                            \item Reverse the columns of $Z_l$: $Z_l = Z_l[:,n-q_2+r:-1:1]$. 
                    \end{itemize} 
                \item To plug \eqref{eq-l-svd} into \eqref{eq-q_2-case2}, we shall update $V$, $L$ and $Z$ accordingly:
                    \begin{itemize}
                        \item Update $V$:
                            \begin{displaymath}
                                V = V\bordermatrix{ & p-q_2 & r & q_2-r \cr
                                    \hfill p-q_2 & I & 0 & 0 \cr
                                    \hfill r & 0 & U_{r} & 0 \cr
                                    \hfill q_2-r & 0 & 0 & I}P^{-1}
                            \end{displaymath}
                        \item Update the first ($r+n-q_2$) columns of $Z$:
                            \begin{displaymath}
                                Z = Z\bordermatrix{ & r+n-q_2 & q_2-r \cr
                                    \hfill r+n-q_2 & Z_{l} & 0 \cr
                                    \hfill q_2-r & 0 & I}
                            \end{displaymath}
                        \item Rewrite $L$ to formulate $S$:
                            \begin{displaymath}
                                S = \bordermatrix{ & n-q_2 & r & q_1+q_2-n-r & n-q_1 \cr
                                    \hfill r & 0 & S_{l}[:,1:r] & 0 & 0 \cr
                                    \hfill q_1+q_2-n-r & 0 & 0 & L_{23} & 0 \cr
                                    \hfill n-q_1 & 0 & 0 & 0 & I \cr
                                    \hfill p-q_2 & 0 & 0 & 0 & 0 }
                            \end{displaymath}
                    \end{itemize}
                \item Since $Z$ is updated, we need to modify $U$ as well:
                        \begin{itemize}
                            \item Set $W$:
                            
                                Let $C_1 = diag(\alpha_{1}, \cdots, \beta_{r+n-q_2})$, $W = C_1Z_{l}$.
                            \item QR decomposition of $W$:
                                \begin{align}
                                    W = Q_{w}R_{w}
                                \end{align}
                            \item Update $U$:
                                \begin{displaymath}
                                    U = U\bordermatrix{ & r+n-q_2 & m+r-l+q_2\cr
                                    \hfill r+n-q_2 & Q_{w} & 0 \cr
                                    \hfill m+r-l+q_2 & I & 0}
                                \end{displaymath}
                        \end{itemize}
            \end{enumerate}
    Putting all the 7 steps together, we have:
    \begin{align}
                Q_1 = UCZ^{T}, \ \ \ \ Q_2 = VSZ^{T}
            \end{align}
            and
            \begin{displaymath}
                C = \bordermatrix{ & n-q_2 & t & n-q_1 \cr
                \hfill n-q_2 & I & 0 & 0 \cr
                \hfill t & 0 & \Sigma_1 & 0 \cr
                \hfill m-q_1 & 0 & 0 & 0}, \  \ \ \
                S = \bordermatrix{ & n-q_2 & t & n-q_1 \cr
                \hfill t & 0 & \Sigma_2 & 0 \cr
                \hfill n-m & 0 & 0 & I \cr
                \hfill p-q_2 & 0 & 0 & 0}
        \end{displaymath}
        where $t = q_1 + q_2 - n$, $\Sigma_1 = diag(\alpha_{l-q_2+1}, \cdots, \alpha_{q_1})$ and $\Sigma_2 = diag(\beta_{l-q_2+1}, \cdots, \beta_{q_1})$.
    \end{enumerate}
    
    \paragraph{Remark} A wide array of algorithms have been proposed to compute the CSD. 
    Among them, LAPACK features an algorithm to compute the CSD of a 2-by-1 partitioned matrix, which is developed by Sutton \cite{sutton2009computing}.  
    
    Given an $(m+p)$-by-$n$ matrix $X$ with orthonormal columns that has been partitioned into a 2-by-1 block structure: 
        \begin{displaymath}
            X = \bordermatrix{ & n  \cr
            \hfill m & X_1 \cr
            \hfill p & X_2}
        \end{displaymath}
    There exist a $m$-by-$m$ matrix $U_1$, an ($p$)-by-($p$) matrix $U_2$, and a $n$-by-$n$ matrix $V_1$ (all are orthogonal) such that:
        \begin{displaymath}
            \begin{pmatrix}
                U_1 & 0 \\
                0 & U_2
            \end{pmatrix}^{T}
            \begin{pmatrix}
                X_1 \\
                X_2
            \end{pmatrix}V_1 = 
            \begin{pmatrix}
                I_1 & 0 & 0 \\
                0 & C & 0 \\
                0 & 0 & 0 \\
                \cline{1-3} 
                0 & 0 & 0 \\
                0 & S & 0 \\
                0 & 0 & I_2
            \end{pmatrix}
        \end{displaymath}
    where $C$ and $S$ are $r$-by-$r$ non-negative diagonal matrices satisfying $C^2 + S^2 = I$, in which $r = min\{m,p,n,m+q-n\}$. $I_1$ is a $k_1$-by-$k_1$ identity matrix and $I_2$ is a $k_2$-by-$k_2$ identity matrix, where $k_1 = max\{n-p,0\}$, $k_2 = max\{n-m,0\}$.
    
    \subsection{Other prominent algorithms}  
    \subsubsection{LAPACK algorithm}\label{LAPACKalg}
    This algorithm \cite[pp.~51--53]{anderson1999lapack} has two phases.  First is a pre-processing step as described in Section \ref{alg}. Next is a Jacobi-style method \cite{paige1986computing} \cite{bai1993computing} to directly compute the GSVD of two square upper trangular matrices, namely, $A_{23}$ and $B_{13}$ in \eqref{eq-alg-1} such that
    \begin{equation} \label{eq-alg-jacobi}
        A_{23} = U_1CRQ_1^{T},\ \ \ \ B_{13} = V_1SRQ_1^{T}.
    \end{equation}
    Here $U_1$, $V_1$ and $Q_1$ are orthogonal matrices, $C$ and $S$ are both real nonnegative matrices satisfying $C^TC + S^TS = I$, $S$ is nonsingular, and $R$ is upper triangular and nonsingular.
    
    \subsubsection{Van Loan's algorithm}
    Golub and Van Loan \cite[pp.~502--503]{golub2013matrix} introduced an algorithm to compute GSVD using CS decomposition for tall, full-rank matrix pairs. 
    
    Assume that $A$ is $m$-by-$n$ and $B$ is $p$-by-$n$ with $m \geq n$ and $p \geq n$, computes an $m$-by-$m$ orthogonal matrix $U$, a $p$-by-$p$ orthogonal matrix $V$, an $n$-by-$n$ nonsingular matrix $X$ and $m$-by-$n$ diagonal matrice $C$, $p$-by-$n$ diagonal matrice $S$ such that $U^{T}AX = C$ and $V^{T}BX = S$. 
    
    \begin{enumerate}[\textit{Step} 1]
        \item Compute the regular QR decomposition of $\begin{pmatrix} A\\ B\end{pmatrix}$:
    
        $\begin{pmatrix}
        A \\ 
        B 
       \end{pmatrix}  = \begin{pmatrix}
        Q_1 \\ 
        Q_2
       \end{pmatrix}R$
       
       \item Compute the CS decomposition of $Q_1$ and $Q_2$:
       
        $U^{T}Q_1Z = C = diag(\alpha_i, \cdots, \alpha_n)$,
        $V^{T}Q_2Z = S = diag(\beta_i, \cdots, \beta_n)$· 
        
        \item Solve $RX = Z$ for $X$.
    \end{enumerate}
    
    \subsection{Justifications on the choice of CS decomposition over Jacobi method}
    